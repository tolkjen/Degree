\documentclass[../thesis.tex]{subfiles}
\begin{document}

\begin{center}
\fontsize{14pt}{18pt}\selectfont
STRESZCZENIE
\end{center}

\begin{flushleft}
Porównanie metod wstępnego przetwarzania i klasyfikacji danych biomedycznych. Celem pracy jest stworzenie systemu komputerowego służącego do porównywania rodzin algorytmów przetwarzania wstępnego i klasyfikacji danych. Aplikacja była testowana na danych dotyczących zachorowalności na raka piersi u kobiet. Praca obejmuje wprowadzenie teoretyczne do zagadnień związanych z uczeniem maszynowym, wymagania postawione systemowi, projekt rozwiązania oraz elementy implementacji. Praca kończy się wnioskami wyciągniętymi z analizy otrzymanych wyników.

\vspace{7.5cm}

Słowa kluczowe: klasyfikacja, przetwarzania wstępne, uczenie maszynowe, obliczenia rozproszone
\end{flushleft}

\noindent\makebox[\linewidth]{\rule{\linewidth}{0.4pt}}

\begin{center}
\fontsize{14pt}{18pt}\selectfont
Comparison of preprocessing and classification methods of biomedical data
\end{center}

\begin{flushleft}
The thesis purpose is to create a computer system comparing families of preprocessing and classification algorithms. The system was tested on women breast cancer data. The thesis includes introduction to machine learning, system requirements, system project and detailed elements of system implementation. The thesis ends with conclusions coming from the analysis of the computed results.

\vspace{7.5cm}

Keywords: classification, preprocessing, machine learning, distributed computing
\end{flushleft}

\thispagestyle{empty}
\cleardoublepage 
\end{document}
