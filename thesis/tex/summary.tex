\documentclass[../thesis.tex]{subfiles}
\begin{document}

\pagestyle{plain}
\chapter{Podsumowanie}

Po zaimplementowaniu systemu komputerowego i przeprowadzeniu na nim badań, stwierdzam, że cele pracy magisterskiej zostały w całości zrealizowane. Projekt systemu, obejmujący architekturę rozwiązania, budowę biblioteki obliczeniowej \emph{mltool} oraz schematy działania pozostałych komponentów, wyraźnie uprościł późniejszą implementację. Użycie deskryptorów algorytmów umożliwiło proste prowadzenie obliczeń w sposób równoległy. 

Przygotowany system komputerowy spełnia założenia pracy. Możliwe jest badanie jakości klasyfikacji dla definiowanych przez użytkownika rodzin algorytmów. System pozwala porównywać rodziny ze sobą, wizualizować rozkład ocen w rodzinach oraz porządkować rodziny wg wybranej miary oceny predykcji. Aplikacja posiada interfejs konsolowy pozwalający na przeprowadzanie wszystkich niezbędnych operacji.

Zrealizowany system obliczeniowy może być w przyszłości ulepszany i rozbudowywany. Modularna budowa systemu nakierowana na obliczenia rozproszone pozwala na łatwe dostosowanie systemu do pracy w klastrze komputerowym lub w chmurze obliczeniowej. Użyte technologie (język programowania \emph{Python}, system kolejkowy \emph{RabbitMQ} oraz biblioteka \emph{Celery}, baza danych \emph{PostgreSQL}) są popularne i aktywnie rozwijane, dzięki czemu zainteresowane osoby mogą z łatwością wprowadzać własne zmiany do systemu.

Badania przeprowadzone w trakcie pisania pracy doprowadziły do wielu wniosków. Po pierwsze, skalowanie danych i ich późniejsza klasyfikacja metodą lasu losowego danej bardzo dobre wyniki w kontekście precyzji oraz miary F1. Po drugie, las losowy oraz drzewa decyzyjne sprawdzają się dobrze dla każdego rodzaju oceny. Kolejną obserwacją jest pozytywny wpływ skalowania atrybutów na działanie drzewa decyzyjnego i naiwnego klasyfikatora bayesowskiego, bez względu na rodzaj oceny predykcji. Badania pozwoliły również odkryć, że usuwanie atrybutów oraz ich grupowanie zazwyczaj prowadzi do pogorszenia oceny. Analizując wyniki obliczeń udało się ustalić, że naiwny klasyfikator bayesowski oraz maszyna wektorów nośnych nie nadają się do pracy z posiadanym zbiorem danych.

Poprawność działania systemu jest do pewnego stopnia zagwarantowana obecnością wielu testów automatycznych. W celu całkowitego upewnienia się co do słuszności wyników, należy przeprowadzić dodatkowe badania dla znanych danych. Wyniki eksperymentów należy porównać z obserwacjami opisanymi w literaturze. Badania tego typu nie zostały przeprowadzone ale są planowane w przyszłości.

Praca magisterska może okazać się przydatna dla osób, które mają zamiar rozpocząć swoją przygodę z bardzo ciekawą dziedziną informatyki, jaką jest uczenie maszynowe, a w szczególności klasyfikacja i przetwarzanie wstępne.

Prezentacja niniejszej pracy była częścią trzydziestego szóstego sympozjum IEEE w Wildze w 2015 roku.

\end{document}
