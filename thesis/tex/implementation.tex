\documentclass[../thesis.tex]{subfiles}
\begin{document}

\pagestyle{plain}
\chapter{Elementy implementacji systemu}

Rozdział ten skupia się na wybranych fragmentach implementacji systemu, które uznałem za szczególnie ciekawe.

\section{Użyte technologie}

Część ta jest poświęcona technologii, w której został zaimplementowany system komputerowy będący jednym z celów pracy magisterskiej.

\subsection{Język programowania Python}

System został w całości zaimplementowany z wykorzystaniem języka programowania \emph{Python}. \emph{Python} jest interpretowanym językiem programowania wysokiego poziomu, o rozbudowanym pakiecie bibliotek standardowych, którego główną ideą jest czytelność i klarowność kodu źródłowego. Jego składnia jest przejrzysta i zwięzła. Python jest dynamicznie typowany i posiada automatyczne zarządzanie pamięcią. Język ten jest zaprojektowany pod kątem wielu paradygmatów programowania: obiektowego (klasy, dziedziczenie), imperatywnego oraz w pewnym stopniu funkcyjnego (praca na listach). Implementacje języka Python są dostępne dla wielu systemów operacyjnych oraz platform sprzętowych, w tym dla systemów wbudowanych.

\begin{figure}[h]
\centering
\lstset{
  frame=single,
  breaklines=true,
  basicstyle=\ttfamily\scriptsize,
  postbreak=\raisebox{0ex}[0ex][0ex]{\ensuremath{\color{red}\hookrightarrow\space}}
}
\lstinputlisting[language=Python]{fib.py}
\caption{Kod funkcji w języku \emph{Python} zwracającej $n$-ty wyraz ciągu Fibonacciego.}
\label{impl:snippet_python}
\end{figure}

Wybrałem ten język programowania, ponieważ znałem go bardzo dobrze i potrafiłem się nim płynnie posługiwać. Drugim, również bardzo istotnym, powodem jest obecność wielu gotowych i przetestowanych bibliotek związanych z algorytmami uczenia maszynowego. Możliwość użycia pewnego, działającego i niezawodnego kodu przesądziła o wyborze tego języka.

Wysoka zwięzłość języka pozwoliła na eksperymentowanie z systemem komputerowym niewielkim kosztem czasu. Wprowadzanie małych zmian i poprawek mogło być realizowane ad-hoc, co znacznie przyspieszyło proces implementacji.

\subsection{System kolejkowy RabbitMQ}

Systemem odpowiedzialnym za przesyłanie komunikatów pomiędzy nadzorcą a robotnikami jest system kolejkowy \emph{RabbitMQ}. Jest to otwarte oprogramowanie umożliwiające bezpieczne umieszczanie wiadomości w kolejce (na serwerze) w celu późniejszego ich odebrania przez zainteresowane procesy. Serwer \emph{RabbitMQ} jest zaimplementowany w języku Erlang i w pełni wspiera otwarty standard komunikacji \emph{AMQP} (ang. \emph{Advanced Message Queuing Protocol}). Twórcy tej technologii chwalą się wysoką wydajnością systemu, łatwością użycia oraz wsparciem ze strony innych firm.

Wybrałem właśnie ten system kolejkowy ze względu na jego przenośność (obsługuje systemy operacyjne z rodziny Windows, Linux i Mac OS) oraz bogate wsparcie dla programistów (obecność bibliotek w wielu językach programowania, w tym w Pythonie).

\subsection{Biblioteka Celery}

W celu uproszczenia kodu związanego z komunikacją z systemem kolejkowym, użyłem gotowej biblioteki o nazwie \emph{Celery}. Biblioteka ta pozwala w prosty sposób zaimplementować elastyczny i niezawodny rozproszony system do obsługi dużej liczby wiadomości (zadań) jednocześnie oferując narzędzia do obsługi takiego systemu. \emph{Celery} udostępnia nieskomplikowany interfejs programistyczny systemu kolejkowego, realizując faktyczną wymianę wiadomości za pomocą innych systemów tj. \emph{RabbitMQ}, \emph{Redis} lub \emph{Mongo DB}. 

\begin{figure}[h]
\centering
\lstset{
  frame=single,
  breaklines=true,
  basicstyle=\ttfamily\scriptsize,
  postbreak=\raisebox{0ex}[0ex][0ex]{\ensuremath{\color{red}\hookrightarrow\space}}
}
\lstinputlisting[language=Python]{celery.py}
\caption{Kod przykładowego programu odbierającego zadania do policzenia sumy liczb (\emph{Celery}). Kod związany z nasłuchiwaniem na nadchodzące zadania obliczeniowe przypomina prostą definicję funkcji w języku \emph{Python}.}
\label{impl:snippet_celery}
\end{figure}

Projekt \emph{Celery} jest przykładem otwartego oprogramowania, jest udostępniany na licencji BSD. Posiada duże zaplecze użytkowników oraz programistów. \emph{Celery} jest biblioteką języka Python.

Użycie \emph{Celery} pomogło utrzymać prostotę i przejrzystość kodu oraz umożliwiło bezproblemową komunikację z systemem \emph{RabbitMQ}.

\subsection{Baza danych PostgreSQL}

Baza \emph{PostgreSQL} jest systemem zarządzania relacyjnymi bazami danych z rozszerzeniami obiektowymi. Jest jednym z trzech najpopularniejszych rozwiązań wolnodostępnych z dziedziny baz danych. System ten jest rozwijany od połowy 1995 roku, całość jest zaimplementowana w języku \emph{C}.

Baza ta została wybrana ze względu na wolny dostęp oraz dużą wydajność.

\subsection{Mapowanie obiektowo-relacyjne SQLAlchemy}

W celu uproszczenia odczytu i zapisu wartości do bazy danych, użyta została biblioteka \emph{SQLAlchemy} udostępniająca warstwę mapowania obiektowo-relacyjnego. Głównym zadaniem mapowania obiektowo relacyjnego jest zapewnienie kontroli nad danymi w bazie za pomocą obiektów występujących w danym języku programowania. \emph{SQLAlchemy} jest biblioteką dla języka \emph{Python}.

Biblioteka \emph{SQLAlchemy} pozwala w prosty sposób dodawać, wyszukiwać i usuwać wiersze z bazy danych. System ten jawnie oddziela operacje przeprowadzane na bazie od samej bazy danych. Możliwe jest zatem użycie tego samego kodu do współpracy z bazą \emph{PostgreSQL}, \emph{MySQL}, \emph{Oracle}, \emph{SQLite} lub nawet\emph{ Microsoft SQL Server}.'


Biblioteka SQLAlchemy została wybrana aby uprościć implementację systemu komputerowego poprzez zdjęcie z programisty odpowiedzialności za prawidłowe konstruowanie kwarend. Z tego punktu widzenia biblioteka spełniła wszystkie oczekiwania.

\subsection{Biblioteka scikit-learn}

Kluczową częścią całego systemu jest implementacja algorytmów przetwarzania wstępnego i klasyfikacji. O ile niektóre fazy przetwarzania są nieskomplikowane i mogą być zaimplementowane własnoręcznie (imputacja, usuwanie atrybutów, skalowanie atrybutów) to przygotowanie kodu odpowiedzialnego za resztę algorytmów (grupowania i klasyfikacji) mogłoby zająć bardzo dużo czasu. Zważywszy, że celem pracy jest porównanie metod, a nie ich implementacja, w systemie została użyta gotowa biblioteka \emph{scikit-learn} zawierająca implementację ,,najtrudniejszych'' algorytmów.

Biblioteka \emph{scikit-learn} jest biblioteką uczenia maszynowego dla języka \emph{Python} rozpowszechnianą na licencji otwartego oprogramowania (licencja BSD). Biblioteka zawiera wiele implementacji popularnych algorytmów klasyfikacji, regresji i grupowania, w tym drzewa decyzyjne, lasy losowe, maszyny wektorów nośnych oraz wiele innych.

Użycie biblioteki \emph{scikit-learn} pozwoliło na błyskawiczne użycie wielu algorytmów opisanych w pracy magisterskiej. Dodatkowo, dokumentacja biblioteki stanowi cenne źródło wiedzy na temat uczenia maszynowego.

\section{Program liczący}

Podrozdział ten koncentruje się na interesujących fragmentach implementacji programu-robotnika, liczącego oceny klasyfikacji dla grup algorytmów.

\subsection{Buforowanie wyników}
\label{impl:opt}

Analizując złożoność czasową każdego etapu algorytmu, można zauważyć, że najwięcej czasu poświęca się na obliczenia związane z grupowaniem oraz klasyfikowaniem danych. O ile klasyfikacja musi być przeprowadzana za każdym razem, o tyle grupowanie niekoniecznie. Dane po grupowaniu (czyli po przetworzeniu wstępnym) można zapamiętać oraz natychmiastowo odtworzyć w następnej iteracji przebiegu badań. Zapamiętywanie informacji o wynikach przetwarzania wstępnego może być zrealizowane poprzez przechowywanie danych w buforze. Bufor taki miałby ograniczoną pojemność, dlatego co jakiś czas należałoby usunąć z niego niepotrzebne wyniki obliczeń. Wyboru tych ,,niepotrzebnych'' można dokonać algorytmem \emph{LRU} (ang. \emph{Least Recently Used}).

\subsubsection{Klasa Cache}

Buforowanie wyników zostało zrealizowane poprzez implementację dwóch klas: \emph{LRUCache} oraz \emph{Cache}. Pierwsza z klas to uniwersalna implementacja bufora opróżnianego zgodnie z algorytmem \emph{LRU}. Druga klasa to bufor dedykowany pod kod związany z ocenianiem algorytmów przetwarzania i klasyfikacji. Klasa ta wewnętrznie używa klasy \emph{LRUCache}. Implementację klasy \emph{Cache} prezentuje listing \ref{impl:snippet_cache}.

\begin{figure}[h]
\centering
\lstset{
  frame=single,
  breaklines=true,
  basicstyle=\ttfamily\scriptsize,
  postbreak=\raisebox{0ex}[0ex][0ex]{\ensuremath{\color{red}\hookrightarrow\space}}
}
\lstinputlisting[language=Python]{cache.py}
\caption{Implementacja klasy \emph{Cache} w języku \emph{Python}.}
\label{impl:snippet_cache}
\end{figure}

Konstruktor klasy \emph{Cache} przyjmuje parametr $n$ oznaczający maksymalny rozmiar bufora. Klasę \emph{Cache} można interpretować jako tablicę asocjacyjną z automatycznym usuwaniem wartości w której kluczem jest trójka wartości $(checksum, split, pd)$, gdzie $checksum$ to suma kontrolna pliku z danymi, $split$ to podział przykładów na zbiór trenujący i testowy, a $pd$ to deskryptor przetwarzania wstępnego.

Metoda \emph{contains} pozwala stwierdzić czy bufor zawiera dany klucz. Metoda \emph{set} służy do ustalenia wartości pod określonym kluczem, a metoda \emph{get} do pobrania wartości spod klucza.

\subsubsection{Klasa LRUCache}

Podstawą optymalizacji opisywanej w tej części rozdziału jest implementacja klasy odpowiedzialnej za algorytm \emph{LRU}.

\begin{figure}[h]
\centering
\lstset{
  frame=single,
  breaklines=true,
  basicstyle=\ttfamily\scriptsize,
  postbreak=\raisebox{0ex}[0ex][0ex]{\ensuremath{\color{red}\hookrightarrow\space}}
}
\lstinputlisting[language=Python]{lru.py}
\caption{Implementacja klasy \emph{LRUCache} w języku \emph{Python}.}
\label{impl:snippet_lru}
\end{figure}

Klasa \emph{LRUCache} przechowuje wewnętrzną kolekcję obiektów typu klucz-wartość. Konstruktor klasy przyjmuje pojedynczy argument $n$ - rozmiar kontenera. Metoda \emph{get} zwraca żądany element, jeżeli istnieje. Jeżeli element nie istnieje, zwracana jest wartość \emph{None}. Metoda add dodaje do bufora nowy element. Jeżeli po dodaniu rozmiar bufora został przekroczony, obiekt klasy \emph{LRUCache} automatycznie usuwa odpowiedni element.

\subsection{Wykonywanie obliczeń}

Zastosowanie biblioteki \emph{Celery} umożliwiło bardzo zwięzłą i przejrzystą implementację programu-robotnika. Dzięki szczegółowemu projektowi rozwiązania, który obejmuje warstwę zadań (deskryptory) oraz modele przetwarzania i klasyfikacji, wynikowy kod jest czytelny i przypomina istotę algorytmu, który reprezentuje (listing \ref{impl:snippet_worker}).

\begin{figure}[h]
\centering
\lstset{
  frame=single,
  breaklines=true,
  basicstyle=\ttfamily\scriptsize,
  postbreak=\raisebox{0ex}[0ex][0ex]{\ensuremath{\color{red}\hookrightarrow\space}}
}
\lstinputlisting[language=Python]{worker.py}
\caption{Ciało funkcji \emph{evaluate} odpowiedzialnej za liczenie ocen klasyfikacji dla bloków algorytmów.}
\label{impl:snippet_worker}
\end{figure}

Rozpoczęcie obliczeń związanych z oceną jakości klasyfikacji bloku algorytmów można traktować jako zdalne wywołanie procedury \emph{evaluate} (\emph{RPC}, ang. \emph{Remote Procedure Call}). Parametrami takiego wywołania są ścieżka do pliku z danymi, obiekt przechowujący ziarno pseudolosowe oraz zbiór par deskryptorów (przetwarzania wstępnego i klasyfikacji).

W pierwszej kolejności, opierając się na liczbie przykładów w pliku oraz na ziarnie pseudolosowym, generowane jest 10 podziałów zbioru danych na zbiory trenujące i testowe (funkcja \emph{get\textunderscore splits()}).

Następnie dla każdej pary i dla każdego podziału zbioru tworzony jest obiekt $t$ klasy \emph{Sample} zawierający tylko te przykłady, które wchodzą w skład zbioru trenującego. Na podstawie tego obiektu tworzony jest model imputacji. Modelu używa się aby usunąć braki z obiektu $t$. Dalej z obiektu usuwane są wybrane atrybuty. Kolejnym krokiem jest utworzenie modelu skalowania używając obiektu $t$. W kolejnym kroku model ten służy do przeskalowania atrybutów obiektu $t$. Kiedy to nastąpi, wytwarzany jest model grupowania. Używa się go następnie do zgrupowania wybranych atrybutów w $t$.

Po wygenerowaniu wszystkich potrzebnych modeli na zbiorze trenującym i po przetworzeniu tego zbioru, przychodzi kolej na transformację zbioru testowego. Poddaje się go kolejno etapom imputacji, usuwania atrybutów, skalowania i grupowania.

Klasyfikacja danych odbywa się po przetworzeniu wstępnym zbioru uczącego i testowego. Najpierw, przy użyciu metody \emph{create\textunderscore classifier()}, tworzony jest wybrany klasyfikator $clf$. Nauka, test i ocena klasyfikatora $clf$ odbywa się przy użyciu funkcji pomocniczej \emph{get\textunderscore scores()}. Wynik funkcji - trójka ocen predykcji - zostaje dodana do zbioru ocen.

Optymalizacja opisana w punkcie \ref{impl:opt} polega na użyciu obiektu klasy \emph{Cache}. Przed rozpoczęciem każdego przetwarzania wstępnego sprawdza się, czy wyniki znajdują się już w buforze. Jeżeli tak, stają się natychmiast dostępne i program przechodzi do oceny klasyfikacji. Jeżeli nie, program wykonuje przetwarzanie i utrwala (chwilowo) wyniki w buforze.

Funkcja \emph{evaluate()} zwraca zbiór ocen, co w przypadku biblioteki \emph{Celery} oznacza wysłanie tego zbioru do programu-nadzorcy.

\end{document}
